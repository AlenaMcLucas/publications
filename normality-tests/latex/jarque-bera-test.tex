\documentclass[preview]{standalone}

\usepackage{xcolor}
\pagecolor{white}
\usepackage{relsize}
\usepackage[none]{hyphenat}

\begin{document}
Step 1: Calculate sample skewness $S$.
{\Large
\[
S=\frac{1}{n}\sum^n_{i=1} \left(\frac{x_i-\bar{x}}{s}\right)^3
\]
}
where:
\begin{itemize}
	\setlength\itemsep{0.1em}
	\item $\bar{x}$ is the sample mean
	\item $s$ is the sample standard deviation
	\item $n$ is the sample size
\end{itemize}
Step 2: Calculate sample kurtosis $K$.
{\Large
\[
K=\frac{1}{n}\sum^n_{i=1} \left(\frac{x_i-\bar{x}}{s}\right)^4
\]
}
Step 3: Calculate the Jarque-Bera test statistic.
{\Large
\[
JB=\frac{n}{6}\left(S^2+\frac{(K-3)^2}{4}\right)
\]
}
where:
\begin{itemize}
	\setlength\itemsep{0.1em}
	\item $S^2$ is the skewness term
	\item $(K-3)^2$ is the kurtosis term, measuring any difference from 3
\end{itemize}
Squaring these terms ensures that negative and positive skewness or kurtosis both contribute equally.
\end{document}