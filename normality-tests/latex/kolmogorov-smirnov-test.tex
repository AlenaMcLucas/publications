\documentclass[preview]{standalone}

\usepackage{xcolor}
\pagecolor{white}
\usepackage{relsize}
\usepackage[none]{hyphenat}

\begin{document}
Step 1: Calculate the EDF, which is a step function that estimates the cumulative distribution of a sample.
{\Large
\[
EDF=F_n(x)=\frac{1}{n}\sum^n_{i=1}I(X_i\le x)
\]
}
where:
\begin{itemize}
	\setlength\itemsep{0.1em}
	\item $I(X_i\le x)$ is the indicator function that is 1 if the sample is less than or equal to $x$ and 0 otherwise
	\item $n$ is the sample size
\end{itemize}
Step 2: Calculate the Kolmogorov-Smirnov test statistic:
{\Large
\[
D_n=sup_x|F_n(x)-F(x)|
\]
}
where:
\begin{itemize}
	\setlength\itemsep{0.1em}
	\item $F_n(x)$ is the EDF
	\item $F(x)$ is the reference CDF
	\item $sup_x$ is the supremum, or in this case, the maximum absolute difference across all values of $x$
\end{itemize}
This test statistic measures the maximum absolute difference between the empirical distribution $F_n(x)$ and the reference cumulative distribution $F(x)$.
\\\\
These steps calculate the raw test statistic, but you may also come across the normalized version $D^*_n=\sqrt{n}D_n$.
\end{document}